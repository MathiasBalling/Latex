\section{Theorems}
\subsection{Differential Operators}
\subsubsection{Gradient}
The gradient of a scalar field is a vector field that points in the direction of the steepest increase of the scalar field.
$$\text{grad }f(x,y,z)=\nabla f(x,y,z)=\frac{ \partial f }{ \partial x } \mathbf{i}+\frac{ \partial f }{ \partial y } \mathbf{j}+\frac{ \partial f }{ \partial z } \mathbf{k}$$
$$\mathbf{F}(x,y,z)=f_x(x,y,z)\mathbf{i}+f_y(x,y,z)\mathbf{j}+f_z(x,y,z)\mathbf{k}$$

\subsubsection{Divergence}
The divergence of a velocity field represents the net flow of fluid out of a small volume in a scalar field.
$$\text{div }\mathbf{F}(x,y,z)=\nabla \cdot \mathbf{F}(x,y,z)=\frac{ \partial f_{1} }{ \partial x } +\frac{ \partial f_{2} }{ \partial y } +\frac{ \partial f_{3} }{ \partial z } $$

\subsubsection{Curl}
The curl or field circulation of the electric field gives the rate of change of the magnetic field.
$$\text{curl }\mathbf{F}(x,y,z)
=\nabla \times \mathbf{F}(x,y,z)
=\left|\begin{array}{ccc}\mathbf{i} & \mathbf{j} & \mathbf{k} \\
\frac{ \partial  }{ \partial x } & \frac{ \partial  }{ \partial y } & \frac{ \partial  }{ \partial z } \\
f_{1} & f_{2} & f_{3} \end{array}\right|$$
$$=\left(\frac{ \partial f_{1} }{ \partial y } -\frac{ \partial f_{2} }{ \partial z } \right)\mathbf{i}+\left(\frac{ \partial f_{3} }{ \partial z } -\frac{ \partial f_{1} }{ \partial x } \right)\mathbf{j}+\left(\frac{ \partial f_{2} }{ \partial x } -\frac{ \partial f_{1} }{ \partial y } \right)\mathbf{k}$$

\subsection{Green's Theorem}
Let $R$ be a regular, closed region in the xy-plane whose boundary, $C$, consists of one or more piecewise smooth, simple closed curves that are positively oriented (counterclock vise) with respect to $R$.
$$\oint_{ C } f_1(x, y) d x+f_2(x, y) d y=\iint_{ R }\left(\frac{\partial f_2}{\partial x}-\frac{\partial f_1}{\partial y}\right) d A$$

\subsection{Stokes' Theorem}
Let $S$ be a piecewise smooth, oriented surface in 3-space, having unit normal field $\widehat{N}$ and boundary $C$ consisting of one or more piecewise smooth, closed curves with orientation inherited from $S$.
$$\oint_{C}F\cdot dr=\iint_{S}\text{curl }F\cdot\widehat{N}dS$$

\subsection{Divergence Theorem}
Let $S$, be a closed piecewise smooth surface, which is the boundary of $V$ with normal $\widehat{N}$ pointing outwards.
$$\oiint_{s}(F \cdot \widehat{N}) d S=\iiint_V \operatorname{div} F d V$$

More variants:
$$\iiint_{D}\text{curl }FdV=-\oiint_{s}(F \times\widehat{N})dS$$
$$\iiint_{D}\text{grad }\phi \,dV=\oiint_{s}\phi \,dS$$

\subsection{Examples}
\subsubsection{Example 1: Div and Curl}
Calculate the divergence and curl of the following vector field:
$$\mathbf{F}=\cos x\,\mathbf{i}-\sin y\,\mathbf{j}+z\,\mathbf{k}.$$

\rule{\textwidth}{0.5pt}

\textbf{Divergence:}
$$\text{div }\mathbf{F}(x,y,z)=\nabla \cdot \mathbf{F}(x,y,z)=\frac{ \partial f_{1} }{ \partial x } +\frac{ \partial f_{2} }{ \partial y } +\frac{ \partial f_{3} }{ \partial z } $$
$$\text{div}F=-\sin(x)-\cos(y)+1$$

\textbf{Curl:}
$$\text{curl}F=\left|\begin{array}{ccc}\mathbf{i} & \mathbf{j} & \mathbf{k} \\ \frac{ \partial  }{ \partial x } & \frac{ \partial  }{ \partial y } & \frac{ \partial  }{ \partial z } \\ f_{x} & f_{y} & f_{z} \end{array}\right|$$
$$=\left(\frac{ \partial f_{z} }{ \partial y } -\frac{ \partial f_{y} }{ \partial z } \right)\mathbf{i}+\left(\frac{ \partial f_{x} }{ \partial z } -\frac{ \partial f_{z} }{ \partial x } \right)\mathbf{j}+\left(\frac{ \partial f_{y} }{ \partial x } -\frac{ \partial f_{x} }{ \partial y } \right)\mathbf{k}$$
$$=(0-0)i+(0-0)j+(0+0)k$$
\subsubsection{Example 2: Green's Theorem}
Using Green's Theorem evaluate $\oint_{e}(x^{2}y)\,d x+(x y^{2})d y,$ clockwise bounded of the region:
$$0\leq y\leq{\sqrt{9-x^{2}}}$$

\rule{\textwidth}{0.5pt}

$$\oint_{ C } f_1(x, y) d x+f_2(x, y) d y=\iint_{ R }\left(\frac{\partial f_2}{\partial x}-\frac{\partial f_1}{\partial y}\right) d A$$
$$f_1(x,y)=x^2y\qquad f_2=xy^2$$
$$\frac{\partial f_1}{\partial y}=x^2 \qquad \frac{\partial f_2}{\partial x}=y^2$$
$$\iint_{ R }\left(y^2-x^2\right) d A$$
But since it is clockwise:
$$-\iint_{ R }\left(y^2-x^2\right) d A=\iint_{ R }\left(x^2-y^2\right) d A$$

Using polar coordinates:
$$y^2=9-x^2\quad\Rightarrow\quad r=3$$
Since $y\geq 0$:
$$0\leq\theta\leq\pi$$
Convert the function to polar:
$$x^2=(r\cos(\theta))^2\qquad y^2=(r\sin(\theta))^2$$
$$\int_0^\pi\int_0^3(r^2\cos^2(\theta)-r^2\sin^2(\theta))rdrd\theta =\int_0^\pi\int_0^3r^3(\cos^2(\theta)-\sin^2(\theta))drd\theta$$
$$=\int_0^\pi\left[\frac{1}{4}r^4(\cos^2(\theta)-\sin^2(\theta))\right]_0^3d\theta$$
$$=\frac{81}{4}\int_0^\pi(\cos^2(\theta)-\sin^2(\theta))d\theta=\frac{81}{4}\int_0^\pi\frac{1+\cos(2\theta)}{2}-\frac{1-\cos(2\theta)}{2} d\theta$$
$$\frac{81}{4}\int_0^\pi\frac{1+\cos(2\theta)}{2}-\frac{1-\cos(2\theta)}{2} d\theta=\frac{81}{4}\int_0^\pi \cos(2\theta) d\theta$$
$$=\frac{81}{4}\left[\frac{\sin(2\theta)}{2}\right]_0^\pi=0-0=0$$


\subsubsection{Example 3: Stokes' Theorem}
Evaluate $\oint F\cdot dr$, where $F=-y^3i+x^3j-z^3k$ and $C$ is the curve of intersection of the cylinder
$x^2+y^2\leq 1$ and the plane $2x+2y+z=3$ oriented to have a counterclockwise projection onto the xy-plane.

\rule{\textwidth}{0.5pt}

$$\oint_{C}F\cdot dr=\iint_{S}\text{curl }F\cdot\widehat{N}dS$$
Find curl F:
$$\text{curl }F=
\begin{vmatrix}
  i&j&k\\
\frac{ \partial  }{ \partial x } & \frac{ \partial  }{ \partial y } & \frac{ \partial  }{ \partial z } \\
-y^3&x^3&-z^3
\end{vmatrix}
=i(0-0)-j(0-0)+k(3x^2-(-3y^2))=3(x^2+y^2)k
$$
$$\widehat{N}=(2i+2j+k)$$
$$\text{curl }F\cdot \widehat{N}=(3(x^2+y^2)k)\cdot (2i+2j+k)=3(x^2+y^2)$$
Using cylindrical coordinates:
$$3(x^2+y^2)=3r^2$$
$$0\leq r\leq 1\qquad 0\leq\theta\leq 2\pi$$
Setup integral:
$$\oint_{C}F\cdot dr=\int_{0}^{2\pi}\int_0^1 3r^2\ rdrd\theta$$
$$=\int_{0}^{2\pi}\left[\frac{3r^4}{4}\right]_0^1d\theta=\int_{0}^{2\pi}\frac{3}{4}d\theta$$
$$=\frac{3}{4}\left[ \theta \right]_0^{2\pi}=\boxed{\frac{3\pi}{2}}$$

\subsubsection{Example 4: Divergence Theorem}
Use the Divergence Theorem to calculate the flux of the given vector field out
of the sphere $s$ with equation $x^2 + y^2 + z^2 = a^2$, where $a > 0$ and
$$\mathbf{F}=(x^{3})\mathbf{i}+(3yz^{2})\mathbf{j}+(3y^2z+x^2)\mathbf{k}$$

\rule{\textwidth}{0.5pt}
$$\oiint_{s}(F \cdot \widehat{N}) d S=\iiint_V \operatorname{div} F d V$$
Find div F:
$$\text{div }\mathbf{F}=3x^2+3z^2+3y^2$$
Use spherical coordinates:
$$dV=\rho^2\sin(\phi)d\phi d\rho d\theta$$
$$x^2 + y^2 + z^2 = a^2=\rho^2$$
Limits for the integral:
$$0\leq \rho \leq a\qquad 0\leq\phi\leq \pi\qquad 0\leq\theta\leq 2\pi$$
Convert function to spherical:
$$3x^2+3z^2+3y^2=3(x^2+y^2+z^2)=3\rho^2$$
$$\int_0^{2\pi}\int_0^{\pi}\int_0^a3\rho^2\rho^2\sin(\phi)d\rho d\phi d\theta$$
$$\int_0^{2\pi}\int_0^{\pi}\sin(\phi)\int_0^a3\rho^4d\rho d\phi d\theta$$
$$=\int_0^{2\pi}\int_0^{\pi}\sin(\phi)\left[\frac{3\rho^5}{5} \right]_0^ad\phi d\theta=\int_0^{2\pi}\int_0^{\pi}\sin(\phi)\frac{3a^5}{5}d\phi d\theta$$
$$=\int_0^{2\pi}\int_0^{\pi}\sin(\phi)\frac{3a^5}{5}d\phi d\theta=\frac{3a^5}{5}\int_0^{2\pi}\int_0^{\pi}\sin(\phi)d\phi d\theta$$
$$=\frac{3a^5}{5}\int_0^{2\pi}\left[-\cos(\phi)\right]_0^\pi d\theta=\frac{3a^5}{5}\int_0^{2\pi}-(-1)-(-1) d\theta$$
$$=\frac{6a^5}{5}\int_0^{2\pi} d\theta$$
$$=\frac{6a^5}{5}\left[\theta\right]_0^{2\pi}=\frac{12a^5\pi}{5}$$
