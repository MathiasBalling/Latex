\section{Double-Integrals}
\subsection{Riemann Sum}
$$\sum_{i=1}^n f(x_i^*, y_i^*) \Delta A_i$$

\[
  I=\iint_D f(x,y) dA\quad \quad \text{where } D \text{ is a region in } \mathbb{R}^2 \text{ and } dA \text{ is }dxdy
\]

\subsection{Double Integrals over General domains}
If $f(x,y)$ is defined and bounded on domain $D$, then $\hat{f}(x,y)$ is zero outside D.
$$\iint_{D}f(x,y)\,d A=\iint_{R}{\hat{f}}(x,y)\,d A$$

\subsection{Iteration of Double Integrals}
If $f(x,y)$ is continuous on the bounded y-simple domain $D$ given by $a\leq x\leq b$ and $c(x)\leq y\leq d(x)$, then:
$$\iint f(x,y)\,d A=\int_a^b dx \int_{c(x)}^{d(x)}f(x,y)dy$$

If $f(x,y)$ is continuous on the bounded x-simple domain $D$ given by $c\leq x\leq d$ and $a(x)\leq y\leq b(x)$, then:
$$\iint f(x,y)\,d A=\int_c^d dx \int_{a(x)}^{b(x)}f(x,y)dy$$
\subsection{Double Integrals in Polar Coordinates}
$$dA=dxdy = r\ dr d\theta$$
$$\begin{array}{lll}
  x=r\cos(\theta)&&r^2=x^2+y^2\\
  y=r\sin(\theta)&&\tan(\theta)=\frac{y}{x}
\end{array}$$
\subsubsection{Limits for Polar Coordinates}
$r$ is the radius from origin to the point
$$r\geq 0$$
$\theta$ is the angle in the positive direction of the xy-plane
$$0\leq\theta \leq 2\pi$$

\subsection{Change of Variables in Double Integrals}
If $x$ and $y$ are given as a function of $u$ and $v$:
$$\begin{array}{c}
x=x(u,v)\\
y=y(u,v)
\end{array}$$
These can be transformed or mapped from points $(u,v)$ in the $uv$-plane to points $(x,y)$ in the $xy$-plane.\\
The inverse transformation is given by:
$$\begin{array}{c}
  u=u(x,y)\\
  v=v(x,y)
\end{array}$$
Scaled area element:
$$dA = dxdy=\left|\frac{\partial(x,y)}{\partial(u,v)}\right|dudv$$
where the Jacobian is:
$$\left|\frac{\partial(x,y)}{\partial(u,v)}\right|=\frac{\partial x}{\partial u}\frac{\partial y}{\partial v}-\frac{\partial x}{\partial v}\frac{\partial y}{\partial u}$$
Let $x(u, v)$ and $y(u,v)$ be a one-to-one transformation from a domain S in the $uv$-plane onto a domain D
$xy$-plane.\\
Suppose, that function $x$ and $y$, and first partial derivatives with respect to $u$ and $v$ are continuous in S.
If $f(x,y)$ is integrable on D, then $g(u,v)=f(x(u,v),y(u,v))$ is integrable on S and:
$$\iint_D f(x,y)dA=\iint_S g(u,v)\left|\frac{\partial(x,y)}{\partial(u,v)}\right|dudv$$

\subsection{Examples}
\subsubsection{Example 1: Change of Variables}
Evaluate the double integral:
$$\iint (x-3y)dA$$
where $R$ is triangular region with vertices $(0,0)$, $(2,1)$, and $(1,2)$ using the transformation:
$$\begin{array}{c}
  x=2u+v\\
  y=u+2v
\end{array}$$
\noindent\rule{\textwidth}{1pt}
As $x$ and $y$ are dependent on $u$ and $v$ the transformation of $dA$ is given by 
$$dA = dxdy=\left|\frac{\partial(x,y)}{\partial(u,v)}\right|dudv$$
Using:
$$\left|\frac{\partial(x,y)}{\partial(u,v)}\right|=\frac{\partial x}{\partial u}\frac{\partial y}{\partial v}-\frac{\partial x}{\partial v}\frac{\partial y}{\partial u}$$
the Jacobian can be calculated:
\begin{eqnarray*}
  \frac{\partial x}{\partial u}=2&
  \dfrac{\partial y}{\partial v}=2\\
  \frac{\partial x}{\partial v}=1&
  \dfrac{\partial y}{\partial u}=1
\end{eqnarray*}
$$2*2-1*1=3$$

Find the boundaries for the double integral\\
$y_1: (0,0) \to (2,1)$ is $y=\frac{1}{2}x$\\
$y_2: (0,0) \to (1,2)$ is $y=2x$\\
$y_3: (1,2) \to (2,1)$ is $y=3-x$

Replace $x$ and $y$ with their transformation:\\
$u+2v=\frac{2u+v}{2}\qquad v=0$\\
$u+2v=4u+2v\qquad u=0$\\
$u+2v=3-2u-v\qquad u=1-v$

Therefore $0\leq u\leq 1-v$ and $0\leq v\leq 1$\\
Now transform the original function $x-3y$:
\begin{eqnarray*}
  x-3y&=&2u+v-3(u+2v)\\
      &=&2u-3u+v-6v\\
      &=&-u-5v
\end{eqnarray*}

\begin{eqnarray*}
  \int_{0}^{1}\int_{0}^{1-v}(-u-5v)\cdot3\ dudv&=&-3 \int_{0}^{1} \left[\frac{u^2}{2}+5 u v\right]^{1-v}_0dv\\
                                               &=&-\frac{3}{2}\int_0^1\left(\frac{27 v^2}{2}-12 v\right)dv\\
                                               &=&\left[\frac{9 v^3}{2}-6 v^2-\frac{3 v}{2}\right]_0^1\\
                                               &=&\boxed{-3}
\end{eqnarray*}
\subsubsection{Example 2: Double integral}
Evaluate the double integral by iteration
$$\iint_{R}(x^{2}+y^{2})\,d A$$
where $R$ is the rectangle $0\leq x\leq a,\,0\leq y\leq b$

\noindent\rule{\textwidth}{1pt}

Insert the limits and solve the integral:
\begin{eqnarray*}
  \int_{0}^{b}\int_{0}^{a}(x^2+y^2)dxdy&=&\int_{0}^{b}\left[\frac{x^3}{3}+x y^2\right]^a_0dy\\
                                       &=&\int_{0}^{b}\left(\frac{a^3}{3}+a y^2\right)dy\\
                                       &=&\boxed{\frac{a^3 b}{3}+\frac{a b^3}{3}}
\end{eqnarray*}

\subsubsection{Example 3: By iteration}
Evaluate the double integral by iteration:
$$\iint_{D}x\cos y\,d A$$
where $D$ is the finite region in the first quadrant bounded by the coordinate axes and the curve
$y=1-x^2$.

\noindent\rule{\textwidth}{1pt}

Given the region the minimum for $x$ and $y$ must be $0$ and for $x$ the maximum is 1:

\begin{eqnarray*}
  \int_{0}^{1}\int_{0}^{1-x^2}(x\cos(y))dydx&=&\int_{0}^{1}x \sin \left(1-x^2\right)dx\\
                                            &=&\left[\frac{1}{2} \sin (1) \sin \left(x^2\right)+\frac{1}{2} \cos (1) \cos \left(x^2\right)\right]^1_0\\
                                            &=&\boxed{\sin ^2\left(\frac{1}{2}\right)}
\end{eqnarray*}

\subsubsection{Example 4: Polar coordinates}

\noindent\rule{\textwidth}{1pt}


