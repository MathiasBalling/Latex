\section{Surface-Integrals}
\subsection{Parametric Surface}
For curve parametrization:
$$r=r(t)=x(t)\mathbf{i}+y(t)\mathbf{j}+z(t)\mathbf{k}\qquad{\mathrm{~where~}}a\leq t\leq b$$
For surface parametrization:
$$r=r(u,v)=x(u,v)\mathbf{i}+y(u,v)\mathbf{j}+z(u,v)\mathbf{k}\qquad{\text{where }}a\leq u\leq b,\quad c\leq v\leq d$$
\subsection{Surface Area}
For a surface the area is given by:
$$\iint_{S}f(x,y,z)d S$$
$$dS=\left|\frac{ \partial r }{ \partial u } \times \frac{ \partial r }{ \partial v } \right|d u d v
=\sqrt{ \left(\frac{ \partial (y,z) }{ \partial (u,v) } \right)^{2}+\left(\frac{ \partial (z,x) }{ \partial (u,v) } \right)^{2}+\left(\frac{ \partial (x,y)}{ \partial (u,v) } \right)^{2}}d u d v$$

For a parametrized surface $S$ given by $r=r(u,v)$, where $(u,v)$ is in the domain $D$ in the $uv$-plane, the surface area is given by:
$$\iint_{S}f \ d S=\iint_{D}f(r(u,v))\left|\frac{ \partial r }{ \partial u } \times \frac{ \partial r }{ \partial v } \right|d u d v$$
$$=\iint_{b}f(x(u,v),y(u,v),z(u,v)){\sqrt{\left({\frac{\partial(y,z)}{\partial(u,v)}}\right)^{2}+\left({\frac{\partial(z,x)}{\partial(u,v)}}\right)^{2}+\left({\frac{\partial(x,y)}{\partial(u,v)}}\right)^{2}}}\,d u\,d v$$

For a surface $S$ given by $z=g(x,y)$, where $(x,y)$ is in the domain $D$ in the $xy$-plane, the surface area is given by:
$$\iint_{S}f(x,y,z)d S=\iint_{D}f(x,y,z(x,y))\sqrt{1+\left(\frac{ \partial g(x,y) }{ \partial x } \right)^{2}+\left(\frac{ \partial g(x,y) }{ \partial y } \right)^{2}}d x d y$$ 
The projection of normal vector onto the xy-plane is given by:
$$\cos(\gamma)=\frac{1}{\sqrt{1+\left(\frac{ \partial g(x,y) }{ \partial x } \right)^{2}+\left(\frac{ \partial g(x,y) }{ \partial y } \right)^{2}}}
\qquad \text{hence }dS=\frac{1}{\cos(\gamma)}dxdy
$$
\subsection{Oriented Surface}
\begin{itemize}
  \item A smooth surface $S$ in 3-space is said to be orientable if there exists a unit vector field $\widehat{N}(P)$.
  \item $\widehat{N}(P)$ defined on $S$ that varies continuously as $P$ ranges over $S$ and that is everywhere normal to $S$.
  \item Any such vector field $\widehat{N}(P)$ determines an orientation of S. 
  \item The oriented surface must have two sides.
  \item $\widehat{N}(P)$ can have only one value at each point $P$ with two sides.
\end{itemize}
\subsection{Flux}
$$\mathbf{F}=f_1(x,y,z)\mathbf{i}+f_2(x,y,z)\mathbf{j}+f_3(x,y,z)\mathbf{k}$$
Given any continuous vector field $\mathbf{F}$, flux of $\mathbf{F}$ across the orientable surface $S$ is integral of the normal component of $\mathbf{F}$ over $S$
$$\iint_S\mathbf{F}\cdot dS=\iint_{S}(\mathbf{F}\cdot\mathbf{\widehat{N}})dS$$

If the surface is closed, then the flux is given by:
$$\oiint_S\mathbf{F}\cdot dS=\oiint_{S}(\mathbf{F}\cdot\mathbf{\widehat{N}})dS$$

If $S$ is a parametrized surface given by $r=r(u,v)$, where $(u,v)$ is in the domain $D$ in the $uv$-plane, then the flux is given by:
$$\iint_{S}\mathbf{F}\cdot d S=\iint_{B}\mathbf{F}\cdot\left({\frac{\partial r}{\partial u}}\times{\frac{\partial r}{\partial v}}\right)\,d u\,d v$$
$$=\iint_D\left(f_1\frac{\partial (y,z)}{\partial (u,v)}+f_2\frac{\partial (z,x)}{\partial (u,v)}+f_3\frac{\partial (x,y)}{\partial (u,v)}\right)du\ dv$$

For a surface $S$ given by $z=g(x,y)$, where $(x,y)$ is in the domain $D$ in the $xy$-plane, the flux is given by:
$$\iint_{S}\mathbf{F}\cdot d S=\iint_D\left(-f_{1}\frac{\partial z}{\partial x}-f_{2}\frac{\partial z}{\partial y}+f_{3}\,\right)d x\,d y$$
\subsection{Examples}
\subsubsection{Example 1: Surface area}
Find $\iint_{\mathcal{S}}x\ dS$ over the part of the parabolic cylinder $z=x^2/2$ that lies inside the first octant 
part of the cylinder $x^2+y^2=1$.

\rule{\textwidth}{0.5pt}

Since $z=g(x,y)$ is a function of $x$ and $y$:
$$\iint_{S}f(x,y,z)d S=\iint_{D}f(x,y,z(x,y))\sqrt{1+\left(\frac{ \partial g(x,y) }{ \partial x } \right)^{2}+\left(\frac{ \partial g(x,y) }{ \partial y } \right)^{2}}d x d y$$ 
Find the length:
$$\frac{\partial}{\partial x}(x^2/2)=x$$
$$\frac{\partial}{\partial y}(x^2/2)=0$$
$$\sqrt{1+\left(\frac{ \partial g(x,y) }{ \partial x } \right)^{2}+\left(\frac{ \partial g(x,y) }{ \partial y } \right)^{2}}=\sqrt{1+(2x)^2}$$
Using $r^2=x^2+y^2$ we know that $x$ and $y$ must between 1 and 0:
$$y=\sqrt{1-x^2}$$
Setup integral:
$$\int_0^1\int_0^{\sqrt{1-x^2}}x\sqrt{1+x^2}dydx$$
$$=\int_0^1x\sqrt{1+x^2}\sqrt{1-x^2}dx=\int_0^1x\sqrt{1-x^4}dx$$
Using table lookup:
$$\left[\frac{1}{4} x^2 \sqrt{1-x^4}-\frac{1}{4} \tan ^{-1}\left(\frac{\sqrt{1-x^4}}{x^2}\right)\right]_0^1=\frac{\pi}{8}$$

\subsubsection{Example 2: Flux}
Find the flux of $F=x\mathbf{i}+x\mathbf{j}+\mathbf{k}$ upward through the part of the surface $z=x^2-y^2$
inside the cylinder $x^2+y^2=a^2$

\rule{\textwidth}{0.5pt}

For a surface $S$ given by $z=g(x,y)$, where $(x,y)$ is in the domain $D$ in the $xy$-plane, the flux is given by:
$$\iint_{S}\mathbf{F}\cdot d S=\iint_D\left(-f_{1}\frac{\partial z}{\partial x}-f_{2}\frac{\partial z}{\partial y}+f_{3}\,\right)d x\,d y$$
$$\frac{\partial z}{\partial x}=2x\qquad\frac{\partial z}{\partial y}=-2y$$
Setup integral:
$$\iint (-x(2x)-x(-2y)+1)dxdy=\iint (-2x^2+2yx+1)dxdy$$
Using $r^2=x^2+y^2$ we know the radius is $a$:
$$0\leq r\leq a\qquad 0\leq \theta\leq 2\pi$$
$$dxdy=rdrd\theta$$
$$x=r\cos(\theta)$$
$$y=r\sin(\theta)$$
$$\int_0^{2\pi}\int_0^{a} (-2(r\cos(\theta))^2+2(r\sin(\theta))(r\cos(\theta))+1)rdrd\theta$$
$$=\int_0^{2\pi}-\frac{1}{2} a^4 \cos ^2(\theta)+\frac{1}{2} a^4 \sin (\theta) \cos (\theta)+\frac{a^2}{2}d\theta$$
$$=-\frac{1}{2} \pi  a^2 \left(a^2-2\right)$$
ans:$\frac{\pi}{2}a^{2}(2-a^{2})$

\subsubsection{Example 3: Flux (Parametrized Surface)}
Find the flux of $F=2x\mathbf{i}+y\mathbf{j}+z\mathbf{k}$ upward through the surface $r=u^2v\mathbf{i}+uv^2\mathbf{j}+v^3\mathbf{k}$ where $(0\leq u\leq1,0\leq v\leq1)$ 

\rule{\textwidth}{0.5pt}

ans:$\frac{1}{6}$

% \subsubsection{Example 4: Flux}
%
% \rule{\textwidth}{0.5pt}
%
% \subsubsection{Example 5: Flux}
%
% \rule{\textwidth}{0.5pt}
